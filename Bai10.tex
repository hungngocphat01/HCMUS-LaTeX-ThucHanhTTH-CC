\documentclass[main.tex]{subfiles}
\begin{document}
\section{Bài 10}
\subsection{Câu a}
Giả sử ta gọi bàn cờ có kích thước $n\times m$ đã cho là $B$.
\newcommand{\empt}{\hphantom{1pt}}

\begin{figure}[H]
    \newcommand{\0}{\textcolor{red}{0}}
    \centering
    \begin{tikzpicture}
        \matrix[matrix of nodes,nodes={draw=gray, anchor=center, minimum size=1cm}, column sep=-\pgflinewidth, row sep=-\pgflinewidth] (A) {
            \empt & \empt & \empt & X \\ 
            \empt & \empt & \empt & \empt \\
            \empt & \empt & \empt & \empt \\
            \empt & \empt & \empt & \empt \\
            \empt & \empt & \empt & \empt \\};
        
            \draw[latex-latex] ([yshift=-0.5cm]A.south west) -- ([yshift=-0.5cm]A.south east) node[midway,fill=white] {m};

            \draw[latex-latex] ([xshift=-0.5cm]A.south west) -- ([xshift=-0.5cm]A.north west) node[midway,fill=white] {n};
        \end{tikzpicture}
        \caption*{(B)}
\end{figure}

Giả sử ta bỏ đi góc trên cùng bên phải. Khi đó hàng đầu tiên của bàn cờ còn lại $m-1$ ô. Ta gọi $B'$ là bàn cờ tạo thành từ việc bỏ đi hàng và cột chứa ô đó và $B_1$ là bàn cờ tạo thành từ việc bỏ đi ô đó, ta có:

\begin{figure}[H]
    \newcommand{\0}{\textcolor{red}{0}}
    \centering
    \begin{minipage}{0.49\textwidth}
        \centering
        \begin{tikzpicture}
        \matrix[matrix of nodes,nodes={draw=gray, anchor=center, minimum size=1cm}, column sep=-\pgflinewidth, row sep=-\pgflinewidth] (A) {
            \empt & \empt & X \\ 
            \empt & \empt & \empt & \empt \\
            \empt & \empt & \empt & \empt \\
            \empt & \empt & \empt & \empt \\
            \empt & \empt & \empt & \empt \\};
        
            \draw[latex-latex] ([yshift=-0.5cm]A.south west) -- ([yshift=-0.5cm]A.south east) node[midway,fill=white] {m};
            \draw[latex-latex] ([xshift=-0.5cm]A.south west) -- ([xshift=-0.5cm]A.north west) node[midway,fill=white] {n};
        \end{tikzpicture} \\
        $(B_1)$
    \end{minipage}
    \begin{minipage}{0.49\textwidth}
        \centering
        \begin{tikzpicture}
        \matrix[matrix of nodes,nodes={draw=gray, anchor=center, minimum size=1cm}, column sep=-\pgflinewidth, row sep=-\pgflinewidth] (A) {
            \empt & \empt & \empt & \\
            \empt & \empt & \empt & \\
            \empt & \empt & \empt & \\
            \empt & \empt & \empt & \\};
        
            \draw[latex-latex] ([yshift=-0.5cm]A.south west) -- ([yshift=-0.5cm]A.south east) node[midway,fill=white] {m-1};
            \draw[latex-latex] ([xshift=-0.5cm]A.south west) -- ([xshift=-0.5cm]A.north west) node[midway,fill=white] {n-1};
        \end{tikzpicture} \\
        $(B')$
    \end{minipage}


\end{figure}

\begin{equation*}
    R(B, x) = xR(B', x) + R(B_1) \tag{1}
\end{equation*}
\newpage
Tiếp tục xét $B_1$:
Giả sử ta tiếp tục bỏ đi ô được đánh dấu ở hình trên, khi đó hàng đầu tiên của $(B_1)$ còn lại $m-2$ ô, gọi $(B_1')$ là bàn cờ thu được khi bỏ đi cả hàng và cột của ô đó và $(B_2)$ là bàn cờ thu được khi bỏ đi ô đó, ta sẽ thu được:

\begin{figure}[H]
    \newcommand{\0}{\textcolor{red}{0}}
    \centering
    \begin{minipage}{0.49\textwidth}
        \centering
        \begin{tikzpicture}
        \matrix[matrix of nodes,nodes={draw=gray, anchor=center, minimum size=1cm}, column sep=-\pgflinewidth, row sep=-\pgflinewidth] (A) {
            \empt & \empt \\ 
            \empt & \empt & \empt & \empt \\
            \empt & \empt & \empt & \empt \\
            \empt & \empt & \empt & \empt \\
            \empt & \empt & \empt & \empt \\};
        
            \draw[latex-latex] ([yshift=-0.5cm]A.south west) -- ([yshift=-0.5cm]A.south east) node[midway,fill=white] {m};
            \draw[latex-latex] ([xshift=-0.5cm]A.south west) -- ([xshift=-0.5cm]A.north west) node[midway,fill=white] {n};
        \end{tikzpicture} \\
        $(B_2)$
    \end{minipage}
    \begin{minipage}{0.49\textwidth}
        \centering
        \begin{tikzpicture}
        \matrix[matrix of nodes,nodes={draw=gray, anchor=center, minimum size=1cm}, column sep=-\pgflinewidth, row sep=-\pgflinewidth] (A) {
            \empt & \empt & \empt & \\
            \empt & \empt & \empt & \\
            \empt & \empt & \empt & \\
            \empt & \empt & \empt & \\};
        
            \draw[latex-latex] ([yshift=-0.5cm]A.south west) -- ([yshift=-0.5cm]A.south east) node[midway,fill=white] {m-1};
            \draw[latex-latex] ([xshift=-0.5cm]A.south west) -- ([xshift=-0.5cm]A.north west) node[midway,fill=white] {n-1};
        \end{tikzpicture} \\
        $(B_1') \equiv (B')$
    \end{minipage}
\end{figure}

\begin{equation*}
\begin{aligned}
R(B_1, x) &= xR(B_1', x) + R(B_2,x) \\
&=xR(B', x) + R(B_2,x)
\end{aligned}
\tag{2}
\end{equation*}

Thay (2) vào (1), ta được:

\begin{equation*}
R(B,x) = R(B', x) + R(B', x) + R(B_2, x)
\end{equation*}

Tiếp tục xét bàn cờ $(B_2)$ và lại bỏ đi 1 ô ở hàng đầu tiên, ta thu được bàn cờ $(B_2') = (B')$ và $(B_3)$. Lặp lại bước trên cho mỗi $(B_{i+1})$ sinh ra từ việc bỏ 1 ô khỏi bàn cờ $(B_i)$ theo quy tắc trên cho đến khi hàng đầu tiên không còn ô nào, ta thu được biểu thức sau:

\begin{figure}[H]
    \newcommand{\0}{\textcolor{red}{0}}
    \centering
    \begin{minipage}{0.49\textwidth}
        \centering
        \begin{tikzpicture}
        \matrix[matrix of nodes,nodes={draw=gray, anchor=center, minimum size=1cm}, column sep=-\pgflinewidth, row sep=-\pgflinewidth] (A) {
            \empt & \empt & \empt & \empt \\
            \empt & \empt & \empt & \empt \\
            \empt & \empt & \empt & \empt \\
            \empt & \empt & \empt & \empt \\};
        
            \draw[latex-latex] ([yshift=-0.5cm]A.south west) -- ([yshift=-0.5cm]A.south east) node[midway,fill=white] {m};
            \draw[latex-latex] ([xshift=-0.5cm]A.south west) -- ([xshift=-0.5cm]A.north west) node[midway,fill=white] {n-1};
        \end{tikzpicture} \\
        $(B_m)$
    \end{minipage}
    \begin{minipage}{0.49\textwidth}
        \centering
        \begin{tikzpicture}
        \matrix[matrix of nodes,nodes={draw=gray, anchor=center, minimum size=1cm}, column sep=-\pgflinewidth, row sep=-\pgflinewidth] (A) {
            \empt & \empt & \empt & \\
            \empt & \empt & \empt & \\
            \empt & \empt & \empt & \\
            \empt & \empt & \empt & \\};
        
            \draw[latex-latex] ([yshift=-0.5cm]A.south west) -- ([yshift=-0.5cm]A.south east) node[midway,fill=white] {m-1};
            \draw[latex-latex] ([xshift=-0.5cm]A.south west) -- ([xshift=-0.5cm]A.north west) node[midway,fill=white] {n-1};
        \end{tikzpicture} \\
        $(B_m') \equiv (B')$
    \end{minipage}
\end{figure}

\begin{equation*}
    R(B, x) = mxR(B', x) + R(B_m,x) \tag{*}
\end{equation*}

Theo giả thiết của đề bài, $R(B, x) \equiv R_{n,m}$; $R(B',x) \equiv R_{n-1,m-1}$ và $R(B_m,x) \equiv R_{n-1,m}$, thay vào (*) ta có đpcm:
\begin{equation*}
    R_{n,m} = mxR_{n-1,m-1} + R_{n-1,m}
\end{equation*}


\end{document}