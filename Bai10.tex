\documentclass[main.tex]{subfiles}
\begin{document}
\section{Bài 10}
\subsection{Câu a}
Giả sử ta gọi bàn cờ có kích thước $n\times m$ đã cho là $B$.
\newcommand{\empt}{\hphantom{1pt}}

\begin{figure}[H]
    \newcommand{\0}{\textcolor{red}{0}}
    \centering
    \begin{tikzpicture}
        \matrix[matrix of nodes,nodes={draw=gray, anchor=center, minimum size=1cm}, column sep=-\pgflinewidth, row sep=-\pgflinewidth] (A) {
            \empt & \empt & \empt & X \\ 
            \empt & \empt & \empt & \empt \\
            \empt & \empt & \empt & \empt \\
            \empt & \empt & \empt & \empt \\
            \empt & \empt & \empt & \empt \\};
        
            \draw[latex-latex] ([yshift=-0.5cm]A.south west) -- ([yshift=-0.5cm]A.south east) node[midway,fill=white] {m};

            \draw[latex-latex] ([xshift=-0.5cm]A.south west) -- ([xshift=-0.5cm]A.north west) node[midway,fill=white] {n};
        \end{tikzpicture}
        \caption*{(B)}
\end{figure}

Giả sử ta bỏ đi góc trên cùng bên phải. Khi đó hàng đầu tiên của bàn cờ còn lại $m-1$ ô. Ta gọi $B'$ là bàn cờ tạo thành từ việc bỏ đi hàng và cột chứa ô đó và $B_1$ là bàn cờ tạo thành từ việc bỏ đi ô đó, ta có:

\begin{figure}[H]
    \newcommand{\0}{\textcolor{red}{0}}
    \centering
    \begin{minipage}{0.49\textwidth}
        \centering
        \begin{tikzpicture}
        \matrix[matrix of nodes,nodes={draw=gray, anchor=center, minimum size=1cm}, column sep=-\pgflinewidth, row sep=-\pgflinewidth] (A) {
            \empt & \empt & X \\ 
            \empt & \empt & \empt & \empt \\
            \empt & \empt & \empt & \empt \\
            \empt & \empt & \empt & \empt \\
            \empt & \empt & \empt & \empt \\};
        
            \draw[latex-latex] ([yshift=-0.5cm]A.south west) -- ([yshift=-0.5cm]A.south east) node[midway,fill=white] {m};
            \draw[latex-latex] ([xshift=-0.5cm]A.south west) -- ([xshift=-0.5cm]A.north west) node[midway,fill=white] {n};
        \end{tikzpicture} \\
        $(B_1)$
    \end{minipage}
    \begin{minipage}{0.49\textwidth}
        \centering
        \begin{tikzpicture}
        \matrix[matrix of nodes,nodes={draw=gray, anchor=center, minimum size=1cm}, column sep=-\pgflinewidth, row sep=-\pgflinewidth] (A) {
            \empt & \empt & \empt & \\
            \empt & \empt & \empt & \\
            \empt & \empt & \empt & \\
            \empt & \empt & \empt & \\};
        
            \draw[latex-latex] ([yshift=-0.5cm]A.south west) -- ([yshift=-0.5cm]A.south east) node[midway,fill=white] {m-1};
            \draw[latex-latex] ([xshift=-0.5cm]A.south west) -- ([xshift=-0.5cm]A.north west) node[midway,fill=white] {n-1};
        \end{tikzpicture} \\
        $(B')$
    \end{minipage}


\end{figure}

\begin{equation*}
    R(B, x) = xR(B', x) + R(B_1) \tag{1}
\end{equation*}
\newpage
Tiếp tục xét $B_1$:
Giả sử ta tiếp tục bỏ đi ô được đánh dấu ở hình trên, khi đó hàng đầu tiên của $(B_1)$ còn lại $m-2$ ô, gọi $(B_1')$ là bàn cờ thu được khi bỏ đi cả hàng và cột của ô đó và $(B_2)$ là bàn cờ thu được khi bỏ đi ô đó, ta sẽ thu được:

\begin{figure}[H]
    \newcommand{\0}{\textcolor{red}{0}}
    \centering
    \begin{minipage}{0.49\textwidth}
        \centering
        \begin{tikzpicture}
        \matrix[matrix of nodes,nodes={draw=gray, anchor=center, minimum size=1cm}, column sep=-\pgflinewidth, row sep=-\pgflinewidth] (A) {
            \empt & \empt \\ 
            \empt & \empt & \empt & \empt \\
            \empt & \empt & \empt & \empt \\
            \empt & \empt & \empt & \empt \\
            \empt & \empt & \empt & \empt \\};
        
            \draw[latex-latex] ([yshift=-0.5cm]A.south west) -- ([yshift=-0.5cm]A.south east) node[midway,fill=white] {m};
            \draw[latex-latex] ([xshift=-0.5cm]A.south west) -- ([xshift=-0.5cm]A.north west) node[midway,fill=white] {n};
        \end{tikzpicture} \\
        $(B_2)$
    \end{minipage}
    \begin{minipage}{0.49\textwidth}
        \centering
        \begin{tikzpicture}
        \matrix[matrix of nodes,nodes={draw=gray, anchor=center, minimum size=1cm}, column sep=-\pgflinewidth, row sep=-\pgflinewidth] (A) {
            \empt & \empt & \empt & \\
            \empt & \empt & \empt & \\
            \empt & \empt & \empt & \\
            \empt & \empt & \empt & \\};
        
            \draw[latex-latex] ([yshift=-0.5cm]A.south west) -- ([yshift=-0.5cm]A.south east) node[midway,fill=white] {m-1};
            \draw[latex-latex] ([xshift=-0.5cm]A.south west) -- ([xshift=-0.5cm]A.north west) node[midway,fill=white] {n-1};
        \end{tikzpicture} \\
        $(B_1') \equiv (B')$
    \end{minipage}
\end{figure}

\begin{equation*}
\begin{aligned}
R(B_1, x) &= xR(B_1', x) + R(B_2,x) \\
&=xR(B', x) + R(B_2,x)
\end{aligned}
\tag{2}
\end{equation*}

Thay (2) vào (1), ta được:

\begin{equation*}
R(B,x) = R(B', x) + R(B', x) + R(B_2, x)
\end{equation*}

Tiếp tục xét bàn cờ $(B_2)$ và lại bỏ đi 1 ô ở hàng đầu tiên, ta thu được bàn cờ $(B_2') = (B')$ và $(B_3)$. Lặp lại bước trên cho mỗi $(B_{i+1})$ sinh ra từ việc bỏ 1 ô khỏi bàn cờ $(B_i)$ theo quy tắc trên cho đến khi hàng đầu tiên không còn ô nào, ta thu được biểu thức sau:

\begin{figure}[H]
    \newcommand{\0}{\textcolor{red}{0}}
    \centering
    \begin{minipage}{0.49\textwidth}
        \centering
        \begin{tikzpicture}
        \matrix[matrix of nodes,nodes={draw=gray, anchor=center, minimum size=1cm}, column sep=-\pgflinewidth, row sep=-\pgflinewidth] (A) {
            \empt & \empt & \empt & \empt \\
            \empt & \empt & \empt & \empt \\
            \empt & \empt & \empt & \empt \\
            \empt & \empt & \empt & \empt \\};
        
            \draw[latex-latex] ([yshift=-0.5cm]A.south west) -- ([yshift=-0.5cm]A.south east) node[midway,fill=white] {m};
            \draw[latex-latex] ([xshift=-0.5cm]A.south west) -- ([xshift=-0.5cm]A.north west) node[midway,fill=white] {n-1};
        \end{tikzpicture} \\
        $(B_m)$
    \end{minipage}
    \begin{minipage}{0.49\textwidth}
        \centering
        \begin{tikzpicture}
        \matrix[matrix of nodes,nodes={draw=gray, anchor=center, minimum size=1cm}, column sep=-\pgflinewidth, row sep=-\pgflinewidth] (A) {
            \empt & \empt & \empt & \\
            \empt & \empt & \empt & \\
            \empt & \empt & \empt & \\
            \empt & \empt & \empt & \\};
        
            \draw[latex-latex] ([yshift=-0.5cm]A.south west) -- ([yshift=-0.5cm]A.south east) node[midway,fill=white] {m-1};
            \draw[latex-latex] ([xshift=-0.5cm]A.south west) -- ([xshift=-0.5cm]A.north west) node[midway,fill=white] {n-1};
        \end{tikzpicture} \\
        $(B_m') \equiv (B')$
    \end{minipage}
\end{figure}

\begin{equation*}
    R(B, x) = mxR(B', x) + R(B_m,x) \tag{*}
\end{equation*}

Theo giả thiết của đề bài, $R(B, x) \equiv R_{n,m}(x)$; $R(B',x) \equiv R_{n-1,m-1}(x)$ và $R(B_m,x) \equiv R_{n-1,m}(x)$, thay vào (*) ta có đpcm:
\begin{equation*}
    R_{n,m}(x) = mxR_{n-1,m-1}(x) + R_{n-1,m}(x)
\end{equation*}

\subsection{Câu b}

\begin{figure}[H]
    \newcommand{\0}{\textcolor{red}{0}}
    \centering
    \begin{minipage}{0.49\textwidth}
        \centering
        \begin{tikzpicture}
        \matrix[matrix of nodes,nodes={draw=gray, anchor=center, minimum size=1cm}, column sep=-\pgflinewidth, row sep=-\pgflinewidth] (A) {
            \empt & \empt & \empt & \empt \\
            \empt & \empt & \empt & \empt \\
            \empt & \empt & \empt & \empt \\
            \empt & \empt & \empt & \empt \\
            \empt & \empt & \empt & \empt \\};
        
            \draw[latex-latex] ([yshift=-0.5cm]A.south west) -- ([yshift=-0.5cm]A.south east) node[midway,fill=white] {m};
            \draw[latex-latex] ([xshift=-0.5cm]A.south west) -- ([xshift=-0.5cm]A.north west) node[midway,fill=white] {n};
        \end{tikzpicture} \\
        $(B)$
    \end{minipage}
    \begin{minipage}{0.49\textwidth}
        \centering
        \begin{tikzpicture}
        \matrix[matrix of nodes,nodes={draw=gray, anchor=center, minimum size=1cm}, column sep=-\pgflinewidth, row sep=-\pgflinewidth] (A) {
            \empt & \empt & \empt & \\
            \empt & \empt & \empt & \\
            \empt & \empt & \empt & \\
            \empt & \empt & \empt & \\};
        
            \draw[latex-latex] ([yshift=-0.5cm]A.south west) -- ([yshift=-0.5cm]A.south east) node[midway,fill=white] {m-1};
            \draw[latex-latex] ([xshift=-0.5cm]A.south west) -- ([xshift=-0.5cm]A.north west) node[midway,fill=white] {n-1};
        \end{tikzpicture} \\
        $(B')$
    \end{minipage}
\end{figure}

- Gọi bàn cờ có đa thức quân xe $R_{n,m}(x)$ là $(B)$ và bàn cờ có đa thức quân xe $R_{n-1,m-1}$ là $(B')$. \\
- Để dễ chứng minh, giả sử $m \le n$ (nếu $m > n$ thì ta xoay bàn cờ $90^\text{o}$ để điều kiện trên được thoả, đa thức quân xe khi đó vẫn không thay đổi).

\paragraph*{Tính R(B, x)}
\hphantom{1pt}\par
Ta xét $r_k(B)$.\\
Để đặt $k$ quân xe lên bàn cờ $(B)$, ta có $C(n, k)$ cách chọn dòng. Với mỗi dòng, ta lại có $m(m-1)...(m-k)$ cách chọn cột nên ta có:
$$
\begin{aligned}
r_k(B) &= {n\choose k}m(m-1)\dots(m-k) \\
&= {n\choose k}\frac{m!}{(m-k)!}
\end{aligned}
$$

Do ta chỉ có thể đặt tối đa $min(m,n)$ quân xe lên bàn cờ, mà $m\le n$ nên ta có:
$$
R(B, x) = \ssum{k=0}{m}{{n\choose k}\frac{m!}{(m-k!)}x^k}
$$

Lấy đạo hàm cả 2 vế, ta được:
\begin{equation*}
\begin{aligned}
\deriv{R(B,x)} &= \deriv{\ssum{k=0}{m}{{n\choose k}\frac{m!}{(m-k!)}x^k}} \\
&= \ssum{k=1}{m}{k{n\choose k}\frac{m!}{(m-k)!}x^{k-1}} \\
& =\ssum{k=1}{m}{k\frac{m!n!}{(m-k)!(n-k)!k!}x^{k-1}} \\
& =\ssum{k=1}{m}{\frac{m!n!}{(m-k)!(n-k)!(k-1)!}x^{k-1}} \\
& =\ssum{k=1}{m}{\frac{m!n!}{(m-k)!(n-k)!(k-1)!}x^{k-1}} \quad \text{(1)}\\
\end{aligned}
\end{equation*}

\paragraph*{Tính R(B', x)}
\hphantom{1pt}\par 
Lý luận tương tự như với bàn cờ (B), ta có:

\begin{equation*}
\begin{aligned}
R(B',x) &=  \ssum{k=0}{m-1}{{n-1\choose k}\frac{(m-1)!}{(m-1-k!)}x^k} \\
&= \ssum{k=0}{m-1}{\frac{(m-1)!(n-1)!}{(m-1-k)![(n-1-k)]!k!}x^{k}} \\
&= \ssum{k=0}{m-1}{\frac{(m-1)!(n-1)!}{\left[m-(k+1)\right]![n-(k+1)]!k!}x^{k}}
\end{aligned}
\end{equation*}

Nhân $nm$ vào cả 2 vế, ta được:
\begin{equation*}
\begin{aligned}
nmR(B',x) &= \ssum{k=0}{m-1}{nm\frac{(m-1)!(n-1)!}{\left[m-(k+1)\right]![n-(k+1)]!k!}x^{k}}\\
&= \ssum{k=0}{m-1}{\frac{m!n!}{\left[m-(k+1)\right]![n-(k+1)]!k!}x^{k}} \quad \text{(2)}
\end{aligned}
\end{equation*}

Đặt $t = k + 1$, ta có:
\begin{equation*}
\begin{aligned}
& 0 \le k \le m-1 \\
\Ra &\ 1 \le t \le m
\end{aligned}
\end{equation*}

Thay $k = t - 1$ vào (2), ta được:
\begin{equation*}
\begin{aligned}
nmR(B',x) &= \ssum{t=1}{m}{\frac{m!n!}{(m-t)!(n-t)!(t-1)!}x^{t-1}} \quad \text{(3)}
\end{aligned}
\end{equation*}

Trong (3), ta đổi biến $t$ trở lại thành $k$ ta sẽ thu được biểu thức (1).

Vậy ta có:
\begin{equation*}
\begin{aligned}
\deriv{R(B,x)} &=  nmR(B',x) \\
\Ra\ \deriv{R_{n,m}(x)} &= nmR_{n-1,m-1}(x) \quad \text{(đpcm)}
\end{aligned}
\end{equation*}

\end{document}