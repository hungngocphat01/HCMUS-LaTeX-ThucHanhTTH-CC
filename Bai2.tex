\documentclass[main.tex]{subfiles}
\begin{document}
\section{Bài 2}
\subsection{Câu a}
\paragraph*{i.}
Gọi \a{n} là số tiền thu được sau $n$ năm.\\
- Vì sau mỗi năm, số tiền tăng lên 4\% nên so với năm trước, ta có:
\begin{align*}
a_n &= (1+4\%)a_{n-1}\\
&=1,04\ a_{n-1}
\end{align*}
- Do ban đầu ta đã gửi 1000\$ vào sổ nên $a_0=1000$.\\
- Sử dụng Maple, ta tính được:
$$
a_n = 1000\times 1,04^n
$$

\paragraph*{ii.}
Tương tự, do số tiền tăng 4\% sau mỗi năm, và cuối mỗi năm ta lại gửi thêm 100\$ vào tài khoản, nên ta có hệ thức đệ quy:
$$
a_n = 1,04\times a_{n-1}+100
$$
- \textbf{Tìm điều kiện đầu:} do cuối năm đầu tiên ta mới gửi tiền vào, nên sau 1 năm mới có 100\$ trong tài khoản, nên: $a_0=0, \quad a_1=100$.\\
- Sử dụng Maple giải, ta có:
$$
a_n = 2500 \times 1,04^n - 2500
$$

\subsection{Câu b}

\subsection{Câu c}
- Gọi \a{n} là số chuỗi tam phân có độ dài $n$ mà không xuất hiện ``012''.\\
- Ta xét 3 trường hợp sau:

\paragraph{TH1} 
Chuỗi bắt đầu bằng ``1''.\\

\begin{wrapfigure}{L}{0.25\textwidth}
    \begin{tikzpicture}
        \node[draw] (1) {1};
        \node[draw,right=-0.5pt of 1, minimum width = 3cm] {n-1};
    \end{tikzpicture}    
\end{wrapfigure}

Số chuỗi thoả yêu cầu đề bài là số chuỗi tam phâm không có ``012'' tạo bởi $n-1$ kí tự còn lại. \\
Vậy trường hợp này ta có \a{n-1} (chuỗi).\\

\paragraph{TH2} 
Chuỗi bắt đầu bằng ``2''.\\

\begin{tikzpicture}
    \node[draw] (1) {2};
    \node[draw,right=-0.5pt of 1, minimum width = 3cm] {n-1};
\end{tikzpicture}    

Lý luận tương tự, trường hợp này ta có \a{n-1} (chuỗi).\\

\paragraph{TH3} 
Chuỗi bắt đầu bằng ``0''.\\
\subparagraph{Trường hợp tổng quát}
\begin{tikzpicture}
    \node[draw] (1) {0};
    \node[draw,right=-0.5pt of 1, minimum width = 3cm] {n-1};
\end{tikzpicture}    

Lý luận tương tự, trường hợp này ta cũng có \a{n-1} (chuỗi).

\subparagraph{Tuy nhiên} ta thấy được sẽ có 1 trường hợp chuỗi bắt đầu bằng ``012'' nên phải loại nó ra.
\begin{tikzpicture}
    \node[draw] (1) {012};
    \node[draw,right=-0.5pt of 1, minimum width = 3cm] {n-3};
\end{tikzpicture}  
Trường hợp này có \a{n-3} chuỗi.

\paragraph{Vậy tổng kết lại} theo nguyên lý cộng, ta sẽ có
\begin{align*}
a_n &= a_{n-1} + a_{n-1} + a_{n-1} - a_{n-3}\\
&= 3a_{n-1} - a_{n-3}
\end{align*}

\paragraph*{Tìm điều kiện đầu:}
\begin{itemize}
    \item $a_1 = 3^1$ vì có 1 vị trí, mỗi vị trí có 3 cách chọn.
    \item $a_2 = 3^2 = 9$ vì có 2 vị trí, mỗi vị trí có 3 cách chọn.
    \item $a_3 = 3^3-1 = 26$ vì có 3 vị trí, mỗi vị trí có 3 cách chọn. Tuy nhiên, có một trường hợp chuỗi nhận được sẽ là ``012'' nên ta phải trừ nó đi.
\end{itemize}
Mặt khác, $a_3 = 3a_2 - a_0 \Leftrightarrow a_0 = 3a_2 - a_3 = 1$.

Vậy quan hệ đệ quy cần tìm là:
\begin{align*}
a_n &= 3a_{n-1} - a_{n-3} \\
a_0 &= 1,\ a_1 = 3,\ a_2 = 9
\end{align*}

\end{document}