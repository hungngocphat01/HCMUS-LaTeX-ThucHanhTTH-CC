\documentclass[main.tex]{subfiles}
\begin{document}
\section{Bài 3}
\subsection{Câu a}
Sử dụng Maple, ta có tổng cần tìm là:
\begin{verbatim}
> f:=sum(r*(r+1)*(r+2)*(r+3), r=1..n): simplify(f)
\end{verbatim}
$$
\sum^{n}_{r=1}{r(r+1)(r+2)(r+3)}=
\frac{1}{5}n^5 + 2n^4 + 7n^3 + 10n^2 + \frac{24}{5}n
$$

\subsection{Câu b}
- Gọi $e_k$ là số các số nguyên $k$ xuất hiện trong một phân hoạch của 10.\\
- Số phân hoạch của 10 là số nghiệm nguyên của phương trình:
$$
e_1 + 2e_2 + 3e_3 + \dots + 10e_{10} = 10
$$
- Gọi \a{r} là số phân hoạch của $r$. Hàm sinh cho dãy $\left\{a_r\right\}_{0\le r \le 10}$ là:

\begin{align*}
G(x) &= (1+x+x^2+...)(1+x^2+x^4+...)...(1+x^{10}+x^{20}+...)\\
&= \frac{1}{(1-x)(1-x^2)(1-x^3)...(1-x^{10})}
\end{align*}

- Số phân hoạch của $10$ là hệ số của $x^{10}$ trong hàm sinh $G(x)$.\\
- Sử dụng Maple, ta dễ dàng tính được hệ số của $x^{10}$:
\begin{verbatim}
> g:= 1/product(1-x^i, i=1..10)
> coeff(series(g, x, 11), x, 10)
\end{verbatim}
$$
42
$$

Vậy số phân hoạch của 10 là 42.

\end{document}