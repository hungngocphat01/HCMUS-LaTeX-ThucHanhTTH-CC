\documentclass[main.tex]{subfiles}
\begin{document}

\newcommand{\Mod}[1]{\ (\mathrm{mod}\ #1)}
\newcommand{\e}{\equiv}
\newcommand{\td}{\Leftrightarrow}

\section{Bài 11}
- Gọi $x$ là tổng số sinh viên trên sân trường.\\
- Từ giả thiết bài toán, ta có $x$ là nghiệm của hệ phương trình đồng dư sau:
\[
    \begin{cases} \tag{I}
        x \e 0 \Mod{5} \\
        x \e 4 \Mod{7} \\
        x \e 3 \Mod{11} \\
    \end{cases}
\]

Ta xét hệ gồm 2 phương trình sau:
\begin{align*}
\begin{cases} \tag{II}
        x \e 4 \Mod{7} \\
        x \e 3 \Mod{11} \\
\end{cases} 
\end{align*}

$$
\begin{aligned}
\text{Ta có (II)} &\td 
\left\{\begin{aligned}
    x &= 4 + 7t \\
    x &\e 3 \Mod{11} \\
\end{aligned}\right.\\
&\td 4+7t \e 3\Mod{11} \\
&\td 7t \e (-1)\Mod{11}
\end{aligned}
$$

Do $\varphi(11)=10$ và $\left(7, 11\right) = 1\ |\ (-1)$ nên
$$
\begin{aligned}
& 7^{10} \e 1\Mod{11}\\
\Ra&\ 7\times 7^9 \times (-1) \e (-1)\Mod{11} \\
\Ra&\ t \e (-7^9)\Mod{11} \\
\Ra&\ t \e 3\Mod{11} \quad  \text{(do $-7^9$ mod $11$ = 3)}\\
\Ra&\ t = 3 + 11u\\\\
\text{Lại có}&\ x = 4+7t \\ 
\Ra&\ x = 4+7(3+11u) = 25+77u\\
\Ra&\ x \e 25\Mod{77}
\end{aligned}
$$

Vậy (II) $\td x \e 25\Mod{77}$\\
Thay kết quả trên vào (I), ta được:
\[
\text{(I)} \td 
    \left\{\begin{aligned}
        x &\e 25\Mod{77} \\
        x &\e 0 \Mod{5}
    \end{aligned}\right.
\]

Hệ phương trình trên tương đương với:
$$
\begin{aligned}
& \left\{\begin{aligned}
    x &= 5t \\
    x &\e 25 \Mod{77} \\
\end{aligned}\right.\\
&\td 5t \e 25\Mod{77} \\
\end{aligned}
$$

Do $\varphi(77)=60$ và $\left(77, 5\right) = 1\ |\ 25$ nên
$$
\begin{aligned}
& 5^{60} \e 1\Mod{77}\\
\Ra&\ 5\times 5^{59} \times 25 \e 25\Mod{11} \\
\Ra&\ t \e (5^{59} \times 25)\Mod{77} \\
\Ra&\ t \e 5\Mod{77} \quad  \text{(do $(5^{59} \times 25)$ mod $77$ = 5)}\\
\Ra&\ t = 5 + 77u\\\\
\text{Lại có}&\ x = 5t \\ 
\Ra&\ x = 5(5+77u) = 25+385u\\
\Ra&\ x \e385\Mod{25}\\
\end{aligned}
$$

Vậy nghiệm của hệ phương trình đồng dư (I) là:
$$
\begin{aligned}
&x \e385\Mod{25}\\
\text{Hay }& x = 25+385u \ \text{($u\in\mathbb{Z}$)} \quad (*) 
\end{aligned}
$$

Do khuôn viên sân trường rộng $100m \times 100m$ và được chia thành nhiều ô $20m \times 20m$ nên số lượng ô của cả sân là:
$$
\frac{100}{20} \times \frac{100}{20} = 25 \text{ (ô)}
$$
- Gọi $n$ là số sinh viên trong 1 ô $\Ra x=25n$.\\
- Theo giả thiết:
$$
\begin{aligned}
& 490 \le n \le 500\\
\td&\ 12250 \le 25n = x \le 12500\\
\td&\ 12250 \le 25+385u \le 12500 \quad \text{(từ (*))}\\
\td&\ 31,7 \le u \le 32,4\\
\end{aligned}
$$
- Do $u\in\mathbb{Z}$ nên ta có $u=32$.\\
- Thay giá trị trên vào (*), ta được: 
$$
x = 25+385\times32 = 12345
$$
- Vậy trên sân trường có 12345 sinh viên.



\end{document}